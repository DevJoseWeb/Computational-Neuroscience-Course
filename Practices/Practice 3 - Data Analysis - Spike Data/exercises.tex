\documentclass[a4paper,11pt]{article}
\usepackage[utf8]{inputenc}
\usepackage{algorithmic}
\usepackage{algorithm}
\usepackage{pst-plot}
\usepackage{graphicx}
\usepackage{endnotes}
\usepackage{graphics}
\usepackage{floatflt}
\usepackage{wrapfig}
\usepackage{amsfonts}
\usepackage{amsmath}
\usepackage{verbatim}
\usepackage{hyperref}
\usepackage{multirow}
\usepackage{pdflscape}

\usepackage{hyperref}
\hypersetup{pdfborder={0 0 0 0}}

\pdfpagewidth 210mm
\pdfpageheight 297mm 
\setlength\topmargin{0mm}
\setlength\headheight{0mm}
\setlength\headsep{0mm}
\setlength\textheight{250mm}	
\setlength\textwidth{159.2mm}
\setlength\oddsidemargin{0mm}
\setlength\evensidemargin{0mm}
\setlength\parindent{7mm}
\setlength\parskip{0mm}

\newenvironment{exercise}[3]{\paragraph{Exercise #1: #2 (#3pt)}\ \\}{
\medskip}
\newcommand{\question}[2]{\setlength\parindent{0mm}\ \\$\mathbf{Q_#1:}$ #2\ \\}

\author{\large{Ilya Kuzovkin, Raul Vicente}}
\title{\huge{Introduction to Computational Neuroscience}\\\LARGE{Practice III: Data Analysis - Spiking Data}}

\begin{document}
\maketitle

\begin{exercise}{1}{Questionnaire}{0.5}
\question{1}{...}
\end{exercise}

\begin{exercise}{2}{Spikes}{0.5}
Some theoretical exercise
\end{exercise}

\begin{exercise}{3}{Extract orientation}{1.5}
Some introductory text ...\\
Data description \url{http://crcns.org/files/data/lgn-1/crcns_lgn-1_data_description.pdf}\\

These files contain the spiking responses of LGN neurons in the mouse to 
drifting gratings. Each file contains the spiking responses and stimulus 
specifications for a single neuron. 
 
These files are in the Matlab format. Loading one in MatLab will bring a structure 
called mlgn into MatLab's memory. The structure has two elements: 
 
spktimes: MxNxT size array, where M is the stimulus number, N is the repeat 
number and T is time in milliseconds. Value of 1 in the array indicates 
a spike. Value of 0 indicates no spike. The order of the presentation 
of the stimulus repeats is not provided. 
 
stim: M size array, where M is the number of stimulus conditions. 
stim for this dataset indicates the contrast used, spatial frequency, 
temporal frequency, stimulus duration (in seconds), prestimulus 
duration, poststimulus duration, the original scanrate, orientation, and 
whether muscimol was present in V1. 
\ \\



Data \url{http://crcns.org/data-sets/lgn/lgn-1/about}\\
Article \url{http://www.jneurosci.org/content/33/26/10616.full.pdf}
\begin{enumerate}
\itemsep 0em
	\item Download and understand data
	\item Extract orientation from data
\end{enumerate}
\end{exercise}

\begin{exercise}{4*}{Where is the rat?}{1}
Data \url{http://crcns.org/data-sets/hc/hc-3/about-hc-3}\\
Description \url{http://crcns.org/files/data/hc3/crcns-hc3-data-description.pdf}\\
Where is rat at time moment 12:09?
\end{exercise}

\begin{exercise}{5}{???}{0.5}
Discuss something
\end{exercise}



\ \\
\ \\
\ \\
\ \\
\ \\
Please submit a \texttt{pdf} report with answers to the questions and comments about your solutions. Also submit a code for the programming exercise(s). Pack those into \texttt{zip} archive and upload to the course web page.

\end{document}










