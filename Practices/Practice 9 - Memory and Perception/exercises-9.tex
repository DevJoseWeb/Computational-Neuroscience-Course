\documentclass[a4paper,11pt]{article}
\usepackage[utf8]{inputenc}
\usepackage{algorithmic}
\usepackage{algorithm}
\usepackage{pst-plot}
\usepackage{graphicx}
\usepackage{endnotes}
\usepackage{graphics}
\usepackage{floatflt}
\usepackage{wrapfig}
\usepackage{amsfonts}
\usepackage{amsmath}
\usepackage{amssymb}
\usepackage{verbatim}
\usepackage{hyperref}
\usepackage{multirow}
\usepackage{pdflscape}
 \usepackage{enumitem}

\usepackage{hyperref}
\hypersetup{pdfborder={0 0 0 0}}

\pdfpagewidth 210mm
\pdfpageheight 297mm 
\setlength\topmargin{0mm}
\setlength\headheight{0mm}
\setlength\headsep{0mm}
\setlength\textheight{250mm}	
\setlength\textwidth{159.2mm}
\setlength\oddsidemargin{0mm}
\setlength\evensidemargin{0mm}
\setlength\parindent{7mm}
\setlength\parskip{0mm}

\newenvironment{exercise}[3]{\paragraph{Exercise #1: #2 (#3pt)}\ \\}{
\medskip}
\newcommand{\question}[2]{\setlength\parindent{0mm}\ \\$\mathbf{Q_{#1}:}$ #2\ \\}

\author{\large{Jaan Aru, Ilya Kuzovkin}}
\title{\huge{Introduction to Computational Neuroscience}\\\LARGE{Practice IX: Brain-Computer Interfaces}}

\begin{document}
\maketitle


%
% Intro
%
\ \\

\ \\
\textbf{Request:} Please record the time you will spend of this homework and add it to the report. This is just for me to balance the amount and the difficulty level of the exercises.

%
% 
%
\begin{exercise}{1}{Questionnaire}{0.5}
Please provide full and detailed answers.\\
\question{1}{Explain the difference and the relationship between episodic and semantic memory.}
\question{2}{How does the Morris water maze work?}
\question{3}{Explain one candidate mechanism for working memory.}
\question{4}{Which brain area is crucial for declarative long-term memory?}
\end{exercise}


%
% 
%
\begin{exercise}{2}{Article?}{2}
...
\end{exercise}


%
%  Bonus
%
\begin{exercise}{3}{Backward masking}{1*}
\emph{Backward masking} is a phenomenon which occurs when one visual stimulus (a "mask" or "masking stimulus") is presented immediately after another brief ($\leqslant$ 50 ms) "target" visual stimulus and leads to a failure to consciously perceive the first ("target") stimulus.

\ \\
This page \url{http://www.psych.purdue.edu/~gfrancis/Publications/BackwardMasking} provides simulations for several quantitative models of backward masking.

\ \\
Choose two models and:
\begin{itemize}
	\item Have a look at their code or the paper they were presented in.
	\item Explain how they work.
	\item Discuss their differences.
\end{itemize}

\end{exercise}


%
% Discuss
%
\begin{exercise}{4}{Optical illusions}{0.5}
Your task is to find and explain an example of an optical illusion, which has an explanation of what causes it.
\begin{itemize}
\itemsep 0em
	\item Find a picture of a video with the illusion you find interesting.
	\item Describe what is happening (how our perception is wrong).
	\item Explain what can be the reason behind this illusion, what is the neural mechanism, which prevents us from seeing things as they really are.
	\item How this misperception can affect our everyday life? Why evolution decided to introduce this flaw in our system?
\end{itemize}
Here are some places to check out:
\begin{itemize}
\itemsep 0em
	\item \url{http://michaelbach.de/ot/index.html}
	\item \url{http://illusionoftheyear.com/cat/top-10-finalists}
	\item \url{http://www.youtube.com}
\end{itemize}
\end{exercise}


\ \\
\ \\
\ \\
\ \\
\ \\
Please submit a PDF report with answers to the questions and comments about your solutions. You report should contain figures, explanations, the essential parts of the code you have produced, etc. If the code is too massive you can add it to the submission and upload everything as a \texttt{zip} archive. But single PDF is preferred.

\end{document}










